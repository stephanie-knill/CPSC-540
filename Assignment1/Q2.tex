\section{Convex Functions}
\subsection{Minimizing Stricly Convex Quadratic Functions}
\begin{enumerate}
\item \begin{equation}
f(w) = \frac{1}{2}\| w - v\|^{2}
\end{equation}
Taking the gradient of $f(w)$, we get
\begin{align}
\nabla f(w) &= \nabla \left(\frac{1}{2}(w-v)^{T}(w-v)\right)\\
&=\frac{1}{2}\;\nabla\left(w^{T}w-2w^{T}v +v^{T}v\right)\nonumber\\
&=w-v.\nonumber
\end{align}
We now set the gradient to 0 to find the critical points
\begin{equation}
\nabla f(w_{min}) = 0 = w_{min} - v,
\end{equation}
implies 
\begin{equation}
w_{min} = v.
\end{equation}

\item \begin{equation}
f(w) = \frac{1}{2}\| Xw - y\|^{2}+\frac{1}{2}w^{T}\Lambda w
\end{equation}
Taking the gradient of $f(w)$, we get
\begin{align}
\nabla f(w) &= \nabla \left(\frac{1}{2}\| Xw - y\|^{2}+\frac{1}{2}w^{T}\Lambda w\right)\\
&=\frac{1}{2}\nabla \left[w^{T}\left(X^{T}X+\Lambda\right)w-2w^{T}X^{T}y+y^{T}y\right]\nonumber\\
&=\frac{1}{2}\left(2X^{T}X+\Lambda+\Lambda^{T}\right)w-X^{T}y\nonumber.
\end{align}
Setting the gradient to 0, we find the critical point
\begin{equation}
0 = \frac{1}{2}\left(2X^{T}X+\Lambda+\Lambda^{T}\right)w_{min}-X^{T}y
\end{equation}
which gives,
\begin{equation}
\left(2X^{T}X+\Lambda+\Lambda^{T}\right)w_{min} = 2X^{T}y 
\end{equation}
and finally, 
\begin{equation}
w_{min}=2\left(2X^{T}X+\Lambda+\Lambda^{T}\right)^{-1}X^{T}y
\end{equation}

\item
\begin{equation}
f(w) = \frac{1}{2}\sum_{i=1}^{n} v_{i}\left(w^{T}x_{i} - y_{i}\right)^{2}+\frac{\lambda}{2}\| w-w^{0}\|^{2}
\end{equation}
Taking the gradient of $f(w)$, we get
\begin{align}
\nabla f(w) &= \nabla\left(\frac{1}{2}\sum_{i=1}^{n} v_{i}\left(w^{T}x_{i} - y_{i}\right)^{2}+\frac{\lambda}{2}\| w-w^{0}\|^{2}\right)\\
&=\frac{1}{2}\nabla\left(\sum_{i=1}^{n}v_{i}\left(w^{T}x_{i}x_{i}^{T}w-2w^{T}x_{i}y_{i}+y_{i}^{2}\right)+\lambda\left(w^{T}w-2w^{T}w^{0}+w^{0^{T}}w^{0}\right) \right)\nonumber\\
&= \sum_{i=1}^{n}v_{i}\left(x_{i}x_{i}^{T}w-x_{i}y_{i}\right)+\lambda\left(w-w^{0}\right)\nonumber.
\end{align}
Setting the gradient to 0, we find the critical point
\begin{equation}
0 = \sum_{i=1}^{n}v_{i}\left(x_{i}x_{i}^{T}w_{min}-x_{i}y_{i}\right)+\lambda I\left(w_{min}-w^{0}\right),
\end{equation}
where $I$ is the $d\times d$ identity matrix. Thus,
\begin{equation}
\lambda Iw^{0} + \sum_{i=1}^{n} v_{i}x_{i}y_{i}= \left(\sum_{i=1}^{n}v_{i}x_{i}x_{i}^{T}+\lambda I\right)w_{min}.
\end{equation}
We finally find 
\begin{equation}
w_{min} = \left(\sum_{i=1}^{n}v_{i}x_{i}x_{i}^{T}+\lambda I\right)^{-1}\left(\lambda Iw^{0} + \sum_{i=1}^{n} v_{i}x_{i}y_{i}\right)
\end{equation}
\end{enumerate} 

\subsection{Proving Convexity}
\begin{enumerate}
\item\begin{equation}
f(w) = -\log(aw)
\end{equation}
\par As the log function is twice differentiable, we will prove convexity by calculating the Hessian matrix
\begin{equation}
f^{\prime\prime}(w) = \frac{1}{w^{2}}> 0,\;\;\; \forall\; w\in \mathcal{R}^{+}
\end{equation}
The hessian is therefore positive definite over the domain of $f(w)$ which means f is convex.

\par\item\begin{equation}
f(w) = \frac{1}{2}w^{T}Aw+b^{T}w+\gamma
\end{equation}
Once again, we will use the Hessian matrix criteria,
\begin{equation}
\nabla^{2}f(w) = \frac{1}{2}\left(A+A^{T}\right)\succeq 0.
\end{equation}
As A is positive semi-definite, its transpose is also positive semi-definite and so is the sum $A+A^{T}$.  The Hessian is therefore positive semi-definite, which implies f(w) is convex.

\par\item\begin{equation}
f(w) = \|w \|_{p}
\end{equation}
Here, we will choose two points, $(v,f(v))$ and $(w,f(w))$ and show that the line always lies above the function itself. The line joining the two points can be written parametrically as
\begin{equation}
\theta\left(\sum_{i=1}^{n}\mid w_{i}\mid^{p}\right)^{1/p}+(1-\theta)\left(\sum_{i=1}^{n}\mid v_{i}\mid^{p}\right)^{1/p},\;\;\;\; 0\le\theta \le 1
\end{equation}
while the function in between those two points can be expressed as
\begin{equation}
\left(\sum_{i=1}^{n}\mid \theta w_{i}+(1-\theta)v_{i}\mid^{p}\right)^{1/p}.
\end{equation}
However, the triangle inequality gives us directly that
\begin{align}
f(\theta w+(1-\theta)v)&=\left(\sum_{i=1}^{n}\mid \theta w_{i}+(1-\theta)v_{i}\mid^{p}\right)^{1/p}\\
&\le \theta\left(\sum_{i=1}^{n}\mid w_{i}\mid^{p}\right)^{1/p}+(1-\theta)\left(\sum_{i=1}^{n}\mid v_{i}\mid^{p}\right)^{1/p}\nonumber\\
&=\theta f(w)+(1-\theta)f(v)\nonumber,
\end{align}
showing that the line joining two points is always above the function itself and therefore $f(w)$ is convex.

\par\item\begin{equation}
f(w) = \sum_{i=1}^{n}\log\left(1+\exp\left(-y_{i}w^{T}x_{i}\right)\right)
\end{equation}
Once again, we will compute the Hessian matrix.  First of all, we find the gradient
\begin{align}
\nabla f(w)=\sum_{i=1}^{n}\frac{-y_{i}x_{i}}{1+\exp\left(-y_{i}w^{T}x_{i}\right)}.
\end{align}
The Hessian is therefore
\begin{align}
\nabla^{2} f(w)=\sum_{i=1}^{n}\frac{y_{i}^{2}x_{i}x_{i}^{T}}{\left[1+\exp\left(-y_{i}w^{T}x_{i}\right)\right]^{2}},
\end{align}
which is a real symmetric matrix.  This implies the Hessian is positive semi-definite and the function $f(w)$ is convex.

\par\item\begin{equation}\label{eq:2.2.5}
f(w) = \|Xw-y\|_{p}+\lambda \| Aw\|_{q}
\end{equation}
The first term on the right hand side of equation (\ref{eq:2.2.5}) is simply the affine composition of the $l_{p}-norm$ which we have showed is convex in 3.  That term is therefore convex.  
\par The second term of the expression is the scalar product of the affine composition of the $l_{q}-norm$ and is also convex.
\par The sum of two convex functions is convex and therefore $f(w)$ is convex.

\par\item\begin{equation}
f(w) = \sum_{i=1}^{N}\max\{0,\mid w^{T}x_{i}-y_{i}\mid -\epsilon\}+\frac{\lambda}{2}\|w\|_{2}^{2}
\end{equation}
The second term on the right hand side is convex as it is the scalar product of the $l_{2}-norm$ which is convex.

\par As of the first term on the right hand side, the function $g(w) = 0$ is convex.  The function $h(w) = w^{T}x_{i}-y_{i}$ is convex as it is a linear function.  taking the absolute value of $h(w)$ can be written as $|h(w)| = \max\{-h(w),h(w)\}$ and the maximum of two convex functions is convex.  Substracting a constant $\epsilon$ remains convex as the addition of convex functions is convex.  The maximum of $g(w)$ and $|h(w)|-\epsilon$ is therefore still convex and so is summing over i as it is the sum of convex functions.

\par As both terms on the right hand side are convex and $f(w)$ is the sum of those two terms, $f(w)$ is thus convex.

\par\item\begin{equation}
f(w) = \max_{ijk}\{|x_{i}|+|x_{j}|+|x_{k}|\}
\end{equation}
Once again, the absolute value $|x_{i}| = \max\{x_{i},-x_{i}\}$ is convex as it is the maximum of two convex functions.  The sum $|x_{i}|+|x_{j}|+|x_{k}|$ is convex as it is the sum of convex functions.  The maximum over ijk preserves convexity and therefore $f(w)$ is convex.
\end{enumerate}
